% This is sigproc-sp.tex -FILE FOR V2.6SP OF ACM_PROC_ARTICLE-SP.CLS
% OCTOBER 2002
%
% It is an example file showing how to use the 'acm_proc_article-sp.cls' V2.6SP
% LaTeX2e document class file for Conference Proceedings submissions.
% 
%----------------------------------------------------------------------------------------------------------------
% This .tex file (and associated .cls V2.6SP) *DOES NOT* produce:
%       1) The Permission Statement
%       2) The Conference (location) Info information
%       3) The Copyright Line with ACM data
%       4) Page numbering
%
%  However, both the CopyrightYear (default to 2002) and the ACM Copyright Data
% (default to X-XXXXX-XX-X/XX/XX) can still be over-ridden by whatever the author
% inserts into the source .tex file.
% e.g.
% \CopyrightYear{2003} will cause 2003 to appear in the copyright line.
% \crdata{0-12345-67-8/90/12} will cause 0-12345-67-8/90/12 to appear in the copyright line.
%
%
%---------------------------------------------------------------------------------------------------------------
% It is an example which *does* use the .bib file (from which the .bbl file
% is produced).
% REMEMBER HOWEVER: After having produced the .bbl file,
% and prior to final submission,
% you need to 'insert'  your .bbl file into your source .tex file so as to provide
% ONE 'self-contained' source file.
%
% Questions regarding SIGS should be sent to
% Adrienne Griscti ---> griscti@acm.org
%
% Questions/suggestions regarding the guidelines, .tex and .cls files, etc. to
% Gerald Murray ---> murray@acm.org 
%
% For tracking purposes - this is V2.6SP - OCTOBER 2002


\documentclass[12pt]{article}

\setlength{\oddsidemargin}{0in}
\setlength{\evensidemargin}{0in}
\setlength{\topmargin}{0in}
\setlength{\headheight}{0in}
\setlength{\headsep}{0in}
\setlength{\textwidth}{6in}
\setlength{\textheight}{9in}
\setlength{\parindent}{0in} 

\usepackage{graphicx} %For jpg figure inclusion
\usepackage{times} %For typeface
\usepackage{epsfig}
\usepackage{color} %For Comments
%\usepackage[all]{xy}
\usepackage{float}
%\usepackage{subfigure} 
\usepackage{hyperref}
\usepackage{url}
\usepackage{parskip}

%% Elena's favorite green (thanks, Fernando!)
\definecolor{ForestGreen}{RGB}{34,139,34}
\definecolor{BlueViolet}{RGB}{138,43,226}
\definecolor{Coquelicot}{RGB}{255, 56, 0}
\definecolor{Teal}{RGB}{2,132,130}
%Uncomment this if you want to show work-in-progress comments
\newcommand{\comment}[1]{{\bf \tt  {#1}}}
% Uncomment this if you don't want to show comments
%\newcommand{\comment}[1]{}
\newcommand{\emcomment}[1]{\textcolor{ForestGreen}{\comment{Elena: {#1}}}}
\newcommand{\todo}[1]{\textcolor{blue}{\comment{To Do: {#1}}}}
\newcommand{\tscomment}[1]{\textcolor{Teal}{\comment{Tony: {#1}}}}
%%%%%%%%%%%%%%%%%%%%%%%%%%%%%%%%%%%%%%%%%%

\begin{document}
\pagestyle{plain}
%
% --- Author Metadata here ---
%\conferenceinfo{WOODSTOCK}{'97 El Paso, Texas USA}
%\setpagenumber{50}
%\CopyrightYear{2002} % Allows default copyright year (2002) to be
%over-ridden - IF NEED BE. 
%\crdata{0-12345-67-8/90/01}  % Allows default copyright data
%(X-XXXXX-XX-X/XX/XX) to be over-ridden. 
% --- End of Author Metadata ---

\title{Improving Clojure Error Messages for Programming Novices with clojure.spec}
%\subtitle{[Extended Abstract \comment{DO WE NEED THIS?}]
%\titlenote{}}
%
% You need the command \numberofauthors to handle the "boxing"
% and alignment of the authors under the title, and to add
% a section for authors number 4 through n.
%
% Up to the first three authors are aligned under the title;
% use the \alignauthor commands below to handle those names
% and affiliations. Add names, affiliations, addresses for
% additional authors as the argument to \additionalauthors;
% these will be set for you without further effort on your
% part as the last section in the body of your article BEFORE
% References or any Appendices.

\author{
Myeongjae Song and Elena Machkasova \\
Computer Science Discipline \\
University of Minnesota Morris\\
Morris, MN 56267\\
songx823@morris.umn.edu, elenam@morris.umn.edu
}
\date{}
\maketitle
\thispagestyle{empty}
%\alcomment{Should these say @morris.umn.edu?}

\section*{\centering Abstract}
Functional programming paradigms have been getting mainstream attention recently because of their elegant concurrency 
handling and conciseness of source code. 
University of Minnesota, Morris has a strong tradition of using functional programming languages, in particular those
 in the Lisp family, in the introductory CS curriculum.
A Lisp language is often the very first language that our majors encounter.  
Clojure is a relatively new language in the Lisp family that is becoming increasingly popular nationally 
and worldwide. 
While it would be desirable to use Clojure as a beginner language in CS curriculum, there 
are several challenges in adopting it for novice programmers. 
One of the most significant challenges is its error messages system which uses terminology and concepts
that are unfamiliar to beginners. 
A current project at UMM aims at developing a system of beginner-friendly error messages for Clojure. 
A recently added system of contracts in Clojure, known as clojure.spec, is the evidence that Clojure community 
at large is also interested in improving error reporting.
This system came out in summer 2016, and we have been exploring its ability to facilitate more robust and 
flexible error messages, in particular aimed at helping beginners. 
This paper describes our current system of providing beginner-friendly information in error messages 
and presents our recent work
on switching to clojure.spec for this purpose. 

\newpage
\setcounter{page}{1}

\section{Introduction}
Clojure is a relatively new programming language in the Lisp family, introduced by a developer Rich Hickey in 2007~\cite{Hickey:2008}. 
It quickly gained a worldwide popularity in industry and has a large and very active community of developers. 
As a Lisp language, Clojure follows functional methodology with an emphasis on immutability, functional abstraction, and recursion -- 
concepts that are essential in CS education, as discussed in~\cite{Felleisen:2004}. 
As such, Clojure could be suitable as a first language taught to computer science students, continuing a rich tradition of MIT Scheme~\cite{Abelson},
on par with the Racket programming language, also in the Lisp category~\cite{htdp}. 

However, there are several issues that make introducing Clojure to beginner programmers quite challenging. 
The most notorious of these issues is the lack of beginner-friendly error messages. 
Our research group at University of Minnesota, Morris (UMM) has been working on developing a system of error messages
that are helpful to beginners. 
As our work on this system was progressing, Clojure has introduced a new feature that would allow us to interface
with Clojure core much easier. 
This feature is a system of contracts for data validation, known as clojure.spec.
Thus our current goal is to transition our older ad-hoc system of collecting information about an error 
to clojure.spec. 
This paper presents a comparison between our two approaches and demonstrates our current progress 
in this transition, including some challenges that we have overcome. 

The paper proceeds as following: Section~\ref{sec:background} introduces Clojure and clojure.spec, 
Section~\ref{sec:current} presents our system of error processing before clojure.spec, 
Section~\ref{sec:spec-errors} discusses the new approach, and Section~\ref{sec:conclusion} summarizes our 
current progress and discusses future work. 

\section{Background}\label{sec:background}
	This section is an introduction to Clojure and the clojure.spec library. 
	\subsection{Clojure}\label{sec:clojure}
	Clojure is a Lisp-dialect with some distinct features in comparison to other popular languages, such as Java or Python.
	One such feature is the use of prefix notation in Clojure syntax. In Clojure, every instruction or expression is enclosed in parentheses.
	To invoke a function, you need to specify a function name right after the opening parenthesis and then list all arguments before the 
	closing parenthesis. Parentheses are even required for functions that do not take any arguments.  
	A simple addition of 3 and 4 in Clojure looks like: 
	\begin{verbatim}
	(+ 3 4)
	\end{verbatim}
	In Clojure, even \texttt{+} is a just function, not a special operator. This syntax might look odd, but it is very efficient and easy to parse for machines since the system does not need to 
	calculate operator precedence. Moreover, it easily accommodates a variable number of arguments, which means
	you can provide any number of arguments to some functions, such as the {\tt +} function. For instance,  both {\tt (+ 3 4 5 6)}
	or {\tt (+ 3 4 5 6 10 15)} work.
	%\emcomment{Give an example of a variable number of args}\tscomment{Resolved?}
	%\emcomment{Need to explain in words what that means}
	%\tscomment{Resolved I think}
	
	Defining a function is very intuitive in Clojure. Inside a pair of parentheses, you list \texttt{defn}, the function name, 
	the argument list, and the expression.
	An example of a function that multiplies a given number by two looks like:
	\begin{verbatim}
	(defn multiply-by-two [n] (* n 2))
	\end{verbatim}
	In this example:
	\begin{itemize}
	\item {\tt defn} indicates the declaration of a function.
	\item {\tt multiply-by-two} is the function name.
	\item \texttt{[n]} is the argument list: there is only one argument named {\tt n}.
	\item {\tt (* n 2)} is the expression that multiplies n by 2. The result of this expression will be returned from the function. 
	\end{itemize}
	Calling this function with an argument 3 would look like this {\tt (multiply-by-two 3)} and returns 6.
%	The following are the function call and its result. {\tt =>} is just a notation that we are using to show 
%	what the function returns.
%	 \begin{verbatim}
%	(multiply-by-two 3)
%	=> 6
%	\end{verbatim}
%	\emcomment{show the function call and its result. }
%	\tscomment{Resolved?}
	
	Another functional programming feature that Clojure adopted is immutable variables. In Clojure, once you bind a 
	name to a value, you cannot change what the name is bound to.
%	 \emcomment{It's not just that you can't change the value, you cannot change what the variable is bound to. }
%	\tscomment{Resolved?}
	This is the main source of confusion for developers from imperative programming backgrounds since they are used to 
	changing variables to store different values.
	%\emcomment{immutable variables and data structures are actually slightly different, need to elaborate on this}
	%\tscomment{Changed wording}
	The main benefit of using immutable variables is minimizing side effects, which means 
	developers do not need to worry about variables being modified from unanticipated operations. 
	There are mutable variables in Clojure, but they need to be explicitly declared and modified with special syntax.
	Thus, side effects are not as serious concern as in imperative programming languages. 
%	\emcomment{variables are immutable by default; there are mutable variables, and they need to be explicitly declared as such - need to clarify this, 
%	otherwise our discussion of an atom doesn't make sense}
%	\tscomment{resolved}	
	
	Among various data structures that Clojure supports, sequences are unique in 
	the way they behave. In particular, a sequence can be lazy. In a lazy sequence, element(s) do not exist now, but they 
	can potentially be computed later when they are needed. 
%	\emcomment{will be computed later as needed? }
%	\tscomment{Resolved?}
	In other words, lazy sequences are like a set of rules capable of generating as many elements as are needed.
	Such sequences just exist as a lazy sequence type until some functions force them to be evaluated. Because of its laziness, it is possible 
	for a lazy sequence to represent an infinite sequence. The following is a function that returns a lazy sequence representing the 
	Fibonacci sequence. In this case, the starting values are {\tt  n1} and {\tt n2}. Theoretically, it could generate elements of the Fibonacci 
	sequence indefinitely. 
	\begin{verbatim}
	(defn fibonacci [n1 n2]
	   (lazy-seq (cons n1 (fibonacci n2 (+ n1 n2))))
	\end{verbatim}
	
	Another commonly used Clojure data structure is hash maps. It is essentially a collection of key - value pairs. 
	It is very useful when representing data with relationships. Followings are an example of a hash map to 
	represent a college course: course number is paired with a course name. 
	The key and value are separated by a space, and each key - value pair is 
	separated by a comma.
	\begin{verbatim}
	{:csci3501 "Algorithms", :csci2101 "Data Structures"}
	\end{verbatim}
	In this example:
	\begin{itemize}
	\item The curly braces represent that this is a hash map.
	\item {\tt :csci3501} and {\tt :csci2101} are the keys.
	\item {\tt "Algorithms"} and {\tt  "Data Structures"} are the values.
	\end{itemize}
	
	\subsection{Introduction to clojure.spec}\label{sec:intro-spec}
	The cloure.spec library introduced in Clojure 1.9 in May 2016~\cite{spec}, and technically still in development,  is a contract system that allows 
	Clojure developers to specify and validate expected data 
	%\emcomment{I'd say, expected data, not (just) data types}\tscomment{Resolved?} 
	at runtime. In addition to data validation, clojure.spec provides easy-to-parse error messages when the data is invalid. 
	To start using specs, we need to define a spec first. Each spec is just a specification for data.
	You can use \texttt{s/def} to define specs.

	The following are some simple spec definitions with Clojure predicates:
	\begin{verbatim}
	(s/def ::check-int integer?)
	(s/def ::check-function ifn?)
	\end{verbatim}
	You can use the above spec definitions to check data types with {\tt s/valid?}, which returns a boolean.
	If the data satisfies the spec, {\tt s/valid?} returns true. Otherwise, it returns false. For example, 
	{\tt (s/valid? ::check-int 3.5)} will return {\tt false} because {\tt 3.5} is not an integer number.
	
	It is possible to assemble simple specs to create a more complicated spec. For instance, we can write a spec that verifies 
	the argument types of a function by combining \texttt{::check-int} and \texttt{::check-function} with \texttt{s/cat} 
	which stands for concatenation. 
	\begin{verbatim}
(s/def ::check-int-function 
    (s/cat :first-arg ::check-int 
           :second-arg ::check-function))
	\end{verbatim}
	If the above spec is attached to a function {\tt my-function} that takes two arguments, 
	{\tt (my-function 4 +)} will be valid 
	because the first argument {\tt 4} is an integer and the second argument {\tt +} is a function.
	
	One really powerful feature of spec is the use of regular operators such as \texttt{s/cat}, \texttt{s/*}, and \texttt{s/?}.
	\texttt{s/*} represents zero or more of a data pattern, and \texttt{s/?} represents zero or one of a data pattern.
	Using these regular operators, we can describe a variety of data structures, including lazy sequences and nested data. 

	To define input and output specifications for functions, clojure.spec uses \texttt{s/fdef}. Even though it is possible to 
	define the output spec for a function, it is only usable in spec testing. For our purposes, we will be focused on the input 
	verification. To start spec input validation of a function, it should be turned on by \texttt{(stest/instrument `function-name)}. 
	Otherwise, the spec is completely ignored at runtime. Let's take a look at an example of \texttt{s/fdef}. The following 
	is a simplified definition of \texttt{even?} function and a possible function spec for it. Whenever \texttt{even?} function
	is called, the spec is invoked and checks if the argument is an integer.
	
	\begin{verbatim}
	(defn even? [n] (= (mod n 2) 0))
	(s/fdef even?
    	:args (s/cat ::check-int integer?))
	;; turn on the function spec
	(stest/instrument 'even?)
	\end{verbatim}



\section{Clojure error messages}\label{sec:current}
\subsection{Overview of Clojure error messages and our approaches}
Since Clojure compiles to Java bytecodes and runs on the JVM, Clojure error messages are just Java exceptions
and uses Java datatypes and terminology. 
For instance, {\tt (+ 2 true)} is an attempt to add a number and a boolean. 
Since Java does not automatically convert booleans to numbers, this code results in an error:
\begin{verbatim}
ClassCastException java.lang.Boolean cannot be cast 
to java.lang.Number  
clojure.lang.Numbers.add (Numbers.java:128)
\end{verbatim} 
This is quite cryptic for beginners since they are not familiar with a term ``class'' for a datatype,
``casting'' for type conversion, and ``exception'' for an error. Moreover, even if a term ``boolean'' 
has been introduced to them for true/false values, the prefix {\tt  java.lang} does not 
correspond to their experience. Also, note that the class {\tt Number} has a prefix {\tt  clojure.lang}
which is different from {\tt  java.lang} . To worsen the confusion, instead of referring
to the function {\tt +}, the error message refers to the method {\tt add}, and the class 
in which it is defined is {\tt clojure.lang.Numbers}. Those familiar with Java may identify this
method as a static method (mostly based on the naming convention: {\tt Numbers} is pluralized). 
For new Clojure programmers not familiar with Java, however, the message carries very little, if any, information. 
Even if these new programmers memorize common error messages terminology after a while, 
it would still not have any grounding in their experience. 

Clearly, this situation is not acceptable in an introductory computer science class. Thus, one of the directions of our project is to 
provide a set of error messages that would be more consistent and meaningful in the context of 
novice programmers experience. 

Our error messages translate standard Clojure error messages into terms that are less confusing for beginners. 
For example, the above-mentioned expression  {\tt (+ 2 true)} would result in the error message
\begin{verbatim}
In function "+" the second argument "true" must be a number, 
but is a boolean. 
\end{verbatim} 
In this error message it is clear what function is being called, which argument is causing a problem, 
what is expected, and what is given. 

Our approach to replacing standard error messages has two main cases and a default case:
\begin{enumerate}
\item For commonly used functions we provide a direct assertion to check if we are passing the right number and types 
of parameters, and report an error if we do not. These errors are more informative than default errors.
\item For other cases we perform a lookup of an error message in our ``dictionary'' using pattern-matching, 
and replace the wording so that it is more understandable to beginners. 
\item For rare cases that are not listed in the dictionary we have no choice but to report an error as it is. 
\end{enumerate}
For this paper we are focusing on the first case: providing an assertion to check the number and type of the
arguments. We discuss how we have been handling this situation in the past and how using clojure.spec 
allows us to collect the same information as before (and sometimes more) in a way that is less error-prone. 

\subsection{Assertions for common Clojure functions}\label{sec:assert}
The approach to providing assertions to check function arguments that we used prior to 
incorporating clojure.spec was to write our own function with the same name as a standard one, 
provide a precondition for it to check if the arguments are valid, and if they are, call 
the function provided by core Clojure. Here is an example of this approach used for
function {\tt +} that specifies that it can take any number of arguments, but all
of these arguments must be numbers:
\begin{verbatim}
(defn + [& args]
   {:pre [(check-if-numbers? "+" args 1)]}
   (apply clojure.core/+ args))
\end{verbatim} 
In this example:
\begin{itemize}
\item the name of the function is {\tt +}, 
\item the {\tt [\& args]} is the arguments list. The {\tt \&} sign indicates that what follows is
a list of arguments of arbitrary length. 
\item The next line (with {\tt :pre} in curly braces) is the precondition, followed by a list
(in square brackets) of conditions to check. 
\item In this case there is only one condition to check. It is given by our own 
function {\tt check-if-numbers?} 
and passing to it {\tt args} (the arguments passed to {\tt +}, the name of the function we 
are checking, which is "+", and the starting number of the arguments being checked 
(in this case we are checking all of the arguments, so the starting number is {\tt 1}). 
\item  If {\tt check-if-numbers?}  returns true, the body of the overwritten function {\tt +} 
gets executed. If it return false, an assertion error is thrown by the precondition. 
\item The last line {\tt  (apply clojure.core/+ args)} applies the {\tt +} function in 
{\tt clojure.core} to the list of arguments. One can think of {\tt  clojure.core/+} as 
a path to the function {\tt +} in the core package of Clojure. 
\end{itemize}
Clojure preconditions don't record enough information to generate a helpful error message. 
In particular, they don't record the name of the function they are attached to or the argument 
number. Thus our custom-made predicates, such as  {\tt check-if-numbers?}, record all 
necessary information about failing arguments to augment the preconditions. 

Before spec there was no good way of packaging necessary information into a failed assertion 
exception itself, so we used a global variable (known as an {\it atom} in Clojure) to store this
information. The stored information included:
\begin{itemize}
\item The type that the predicate is checking for, such as a number in the case shown above,
\item The actual type of the argument,
\item The value of the argument,
\item The name of the function for which the assertion fails,
\item The number of the argument, such as first, second, etc. 
\end{itemize}
%\emcomment{fix formatting!}
Revisiting the example above, the call {\tt (+ 2 true)}, will then make  the predicate 
\newline
{\tt check-if-numbers?} be called with the function name {\tt "+"}, the arguments list
{\tt (2, true)}, and the starting number of an argument: 1.   
The predicate then checks the first argument in the list, {\tt 2}, and it is indeed a number. 
Then a recursive call is made with the second argument, {\tt true}. The argument number is 
incremented by 1 to indicate that it is a second argument. Since {\tt true} is not a number,
failure information is recorded in the global variable: the function name is {\tt +}, the ``offending
value'' is {\tt true} and its type is {\tt boolean}, the expected type is {\tt number}, and the argument 
number is {\tt 2}. 

After the information is recorded, the predicate returns {\tt false} to indicate that the precondition 
has failed. Because the precondition failed, an exception {\tt AssertionError} is thrown to the 
place of the program that has made the call to {\tt +} with a wrong argument. 
The exception is then caught. Based on  its type {\tt AssertionError}, it is passed to a handling function 
that retrieves the information from the global variable and forms the error message:
\begin{verbatim}
In function "+" the second argument "true" must be a number, 
but is a boolean. 
\end{verbatim} 
The global variable is then ``cleared'', i.e. the information is deleted from it. This is done so that this 
information is not accidentally retrieved later when it would become irrelevant and confusing. 

In addition to {\tt check-if-numbers} function, we have similar predicates to check if an argument is
a function or a sequences or any other type required by the function. Note, however, that there are
not as many type requirements in Clojure as in a statically typed language, such as Java, so less than  
a dozen such predicates are needed. 

While this approach produces a correct error message, it has drawbacks: using a global variable to
pass this information from one part of the problem to another may be problematic in a multithreaded 
program. This process would also be fragile if multiple errors are taking place in a cascading fashion,
i.e. handling one error triggers another. The switch to clojure.spec eliminates potential inconsistencies 
between the global variable and the error being handled. 

\subsection{Lazy sequences and challenges in argument printing}

As mentioned in Section~\ref{sec:clojure}, Clojure has lazy sequences. Lazy sequences are values in 
Clojure, so they can be passed to other functions unevaluated until their elements are needed. 
This means that if I pass a lazy sequence (for instance, one generated by a function {\tt (constantly 5)} that
produces a lazy infinite sequence of 5s) instead of a number, the value will not be evaluated when an 
error is detected, it will just be kept as a lazy sequence. 

However, 
one of the functions that forces an evaluation of  a lazy sequence, is {\tt print}. Thus we have to be careful to not print 
an infinite (or just very long) sequence accidentally when printing the error message. 
We handle this issue with a preview function that evaluates only the first up to 10 elements of the sequence.

Switching from this approach to clojue.spec has created some issues that we will discuss in Section~\ref{sec:approach}

%\tscomment{Spec-specific issues (inline-functions) will be explained in 'Approaches to improve error messages'. I think 
%just mentioning why checking lazy sequences is hard (we need to evaluate some for checking) is sufficient?}

\section{Improving error messages with clojure.spec}\label{sec:spec-errors}
	\subsection{Benefits of using clojure.spec}
	If there is no spec attached to a function, and the function fails,
	it throws an error which has the information about the type of error, error message, and stack 
	trace. The value that caused the error is rarely given in the error message.
	Let's say a user tried to add {\tt 2} to {\tt true}. Because addition only works for numbers, 
	{\tt true} of type boolean is the value that caused the error or the 'failing value' in this case.
	Even when the failing value is given, we still need to parse the specific value out of 
	the error message string. The same problem occurs when it comes to the line number indicating where it failed. 
	Because an accurate line number is not given in the error message, users need to go through the stack trace to find 
	where the error actually happened. We used to handle these issues by providing assertions as we explained in Section~\ref{sec:assert}.

	Unlike default Clojure errors, if a spec-attached function fails, an exception object 
	\newline	
	{\tt clojure.lang.ExceptionInfo} is thrown. 
	This exception object has a hash map of failure information, which can easily be 
	parsed with Clojure {\tt ex-data} function. A simplified hash map of a spec error 
	for the {\tt (even? false)} function call might look like:
	\begin{verbatim}
	{:clojure.spec/problems [:pred integer?, :val false, :in [0]}], 
	 :clojure.spec/args (false), 
	 :clojure.spec.test/caller {:file "spec_error.clj", :line 87}}
	\end{verbatim}
	The stored information included:
	\begin{itemize}
	\item {\tt :clojure.spec/problems} containing the predicate, failing value, and location of the value 
	(the argument number, in this case 0, which indicates that it is the first argument).
	\item {\tt :clojure.spec/args} containing all the arguments of the function.
	\item {\tt :clojure.spec.test/callers} containing the file name and line number.
	\end{itemize}

	One prerequisite to provide user-friendly error messages is having enough information about the function failure. Since 
	default Clojure error does not provide sufficient data such as the name of failing function or the failing value, our 
	research group had to use a global variable to store this information, as we explained in Section~\ref{sec:assert}. 
	With clojure.spec, we no longer need a global variable because the hash map of {\tt clojure.lang.ExeptoinInfo} object 
	has much more information about the 
	failure compared to the default Clojure error. The information that the embedded hash map provides, contains which predicate 
	failed, what the failing value was, and what the number of the argument was (if the function takes multiple arguments). 
	In addition, they can be easily parsed, as we mentioned earlier. 
	
	\subsection{Approaches to improve error messages}\label{sec:approach}
	With clojure.spec, we have used two different approaches to improve error messages. One way is using a function spec. 
	If function specs are attached to the Clojure core functions, the spec error is thrown when some core function 
	call fails. Then, we can simply parse necessary information from the hash map embedded in the spec error, and 
	process it before we show them to the user. For instance, {\tt min} function 
	takes one or more numbers, and returns the smallest number. We can verify the input of the {\tt min} function by providing a 
	function spec like the following:

	\begin{verbatim}
(s/fdef clojure.core/min
    :args (s/cat :check-numbers (s/+ number?)))
	\end{verbatim}
	This function spec definition checks if {\tt clojure.core/min} function takes at least one argument 
	and if all arguments are of type {\tt number}.

	The main advantage of using a function spec is that we do not need to overwrite Clojure core functions. We can simply attach our 
	function specs to the Clojure core functions. It is also very flexible in that it can be turned off if necessary. 
	We eventually decided to use another approach using spec assert, after we 
	discovered some issues with the simpler approach of just using a function spec. 
	One serious issue is that function specs do not work for inline functions, which are a type of function 
	that is inserted by the complier when they are used.
	Many of Clojure core arithmetic functions are inline functions for better performance.
	Beginners often use arithmetic functions, including {\tt +}, {\tt -}, and {\tt *}, 
	so not being able to use specs for them is a huge downside of the function spec. 
	Moreover, function specs have problems dealing with lazy sequences. When a lazy sequence, 
	that has an error inside, is passed to the function as an argument, the Clojure default error is thrown. 
	For example, let's revisit the {\tt even?} function that we explained in Section 2.2. 
	The {\tt even?} function has an attached spec.
	\begin{verbatim}
	(even? (map even? [3 4 true]))
	\end{verbatim}
	
	The function {\tt map} normally takes two arguments, a function and a collection of elements. 
	What it does is simply apply the given function to each of the collection's elements. 
	If there is an expression {\tt (map even? [1 2 3])}, the {\tt even?} function is applied to 1, 2, and 3.
	As a result of its computation, {\tt map} returns a lazy sequence of false, true, and false
	because only 2 is an even number.
	
	Let's go back to the above example using two {\tt even?} functions. 
	First, the argument of outer {\tt even?} is checked. Then, the lazy sequence {\tt (map even? [3 4 true])} is
	evaluated in some internal Clojure spec checking space. The problem is the function spec for {\tt even?} does not exist
	when the inner {\tt (even? true)} is evaluated. Therefore, the default {\tt java.lang.IllegalArgumentException} is thrown.

	Because of the issues mentioned above, our research team is currently using {\tt s/assert} in lieu of the function spec.
	This approach is very similar to our pre-spec approach that uses {\tt :pre} and requires overwriting 
	Clojure core functions. However, we can solve the issues concerning the inline functions in that 
	the {\tt s/assert} checks the overwritten function's input before the call to the core function is made. 
	
	Following is the overwritten {\tt min} with spec assert. The function {\tt do} just evaluates
	the expressions in order, which means {\tt s/assert} is checked prior to the function call to {\tt clojure.core/min}.
	\begin{verbatim}
	(defn min [arg1 & args]
  		(do 
    		(s/assert (s/cat :check-numbers (s/+ number?)) [arg1 args])
    		(apply clojure.core/min arg1 args)))
	\end{verbatim} 
	Just like the function spec, {\tt s/assert} starts validating data once the switch is turned on by {\tt (s/check-asserts true)}
	Because our {\tt s/assert} is embedded inside the overwritten function, lazy sequence checking is not
	an issue anymore as in function spec.
	
	
\section{Conclusions and future work}\label{sec:conclusion}
	\subsection{Conclusions}
	Clojure.spec provides a more appropriate approach for generating meaningful error messages than our
	previous ad-hoc approach. It automatically collects information necessary to provide useful information 
	about causes of errors. It is also significantly more flexible: we can easily control what kind of information
	we present to programmers. This opens a possibility of creating multiple levels of error messages,
	depending on programmers' experience. 

	In the process of working with clojure.spec we have resolved several technical challenges, such as 
	dealing with inlined functions and lazy sequences. 
	There are still a few unresolved issues, in particular related to multiple errors (errors that happen
	when evaluating a failing value for another error).
	However, with clojure.spec being still in development 
	we may just need to wait until they are resolved by the Clojure community. 

	Overall, using clojure.spec has a promise of creating a robust and customizable system of error 
	handling that relies on features embedded in the language itself, rather than in a separate system. 

	\subsection{Future work}
	There are still many challenges to overcome before Clojure can be easily adopted 
	in an introductory CS classroom. One of the future directions of our work is to continue evaluating what 
	error messages are helpful to beginners. Another significant roadblock is a lack of a beginner-friendly IDE 
	for Clojure, despite the effort of the Clojure community to develop one. Another challenge is to figure out which 
	features of Clojure should be available to beginners: another relatively new Clojure feature, known as transducers, 
	makes certain expressions that beginners can write by mistake be syntactically valid, but almost certainly unintended 
	and confusing. Whether it is possible to ``hide'' these  features from beginners, ideally by using clojure.spec,
	 is a subject of future work. 
	
\bibliographystyle{acm}
\bibliography{mics2017}

\end{document}
